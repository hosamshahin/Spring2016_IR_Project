\chapter{Conclusion}\label{ch:conclusion}

\foxComment{Make clear if did both tweets and webpages}

Since our last report, we have made a great deal of progress in terms of getting our prototype ready to work with other teams. We have managed to process all six of the provided small data files. From those files, we have created the first batch of usable, classified data and imported that data into the HBase system for use by other teams in their areas of the project. In the conclusion of our last paper, we mentioned that we were waiting for updates to be done to the cluster, namely updating to a newer version of Spark. This has since been completed as of this week, and we will now be able to move our system from our private virtual machine over to the Hadoop cluster. This is our next step, and one that we aim to have done as soon as possible. Once the system is up and running on the cluster, we will be able to modify it as mentioned previously so that it will directly read to and write from HBase, which will speed up the processing and eliminate the need for someone to manually index the output files.

% Here is the introduction. The next chapter is chapter~\ref{ch:ch2label}.


% a new paragraph


% \section{Examples}
% You can also have examples in your document such as in example~\ref{ex:simple_example}.
% \begin{example}{An Example of an Example}
%   \label{ex:simple_example}
%   Here is an example with some math
%   \begin{equation}
%     0 = \exp(i\pi)+1\ .
%   \end{equation}
%   You can adjust the colour and the line width in the {\tt macros.tex} file.
% \end{example}

% \section{How Does Sections, Subsections, and Subsections Look?}
% Well, like this
% \subsection{This is a Subsection}
% and this
% \subsubsection{This is a Subsubsection}
% and this.

% \paragraph{A Paragraph}
% You can also use paragraph titles which look like this.

% \subparagraph{A Subparagraph} Moreover, you can also use subparagraph titles which look like this\todo{Is it possible to add a subsubparagraph?}. They have a small indentation as opposed to the paragraph titles.

% \todo[inline,color=green]{I think that a summary of this exciting chapter should be added.}