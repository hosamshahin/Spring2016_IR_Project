\chapter{Introduction}\label{ch:introduction}
%\studentComment{Miscommunication with Solr. Also assumed we were doing redundant work on the clean up part.}

The classification team’s goal is to take collections of tweets and webpages and classify them based on their relevance to given classes or topics. Our team fits into the grand scheme of the project by working between the Collection Management team and the Solr team. The Collection Management team will be responsible for taking the raw tweet and webpage data and filtering out any obvious spam, vulgarity, or otherwise unreadable and unwanted content. The Collection Management team will write the cleaned data into a table in our HBase instance. Once this is done, we will attempt to classify each document's relevance to a specific category. We will explore several methods for accomplishing this, starting with the methodology laid out by last year's Classification team \cite{cui2015classification}. Once we have classified the data, we will be able to pass it along to the Solr team by writing the newly classified data back into the primary HBase table as a column family. The Solr team will then be able to use the column family in the indexing of all the tweet and webpage data. Solr will then provide the indexes for the data to the Front End team, allowing anyone to make use to the systems we are creating.
%The Solr team will import our classified data into the Solr system, making it accessible to anyone using the system we are creating.

This paper represents the third interim report from the Classification team working as a part of the larger scale project for CS5604 Information Storage and Retrieval. The current state of the work is a partial draft of the final report; therefore expect further additions and details in some chapters as the course progresses.

We begin by documenting our understanding of some of the pertinent literature, namely the course textbook and the Classification team report from last year; this can be found in Chapter \ref{ch:literatureReview}, our literature review. Chapter \ref{ch:ReqDesignImp} is the primary section of the document and includes our discussion of the project requirements, an outline for our design for the classification portion of the larger project, and finally a breakdown of our current progress and future plans for the text classification project.

These chapters are then followed by a User Manual in Chapter \ref{ch:userManual}, where we will discuss details in which a user of our methods and programs would have interest. We then include a separate Developer Manual in Chapter \ref{ch:developerManual}, which will document and include details of the code base so that it might be extended and leveraged by other developers.

Following this we include our conclusions about the state of the project thus far and present our thoughts for future work in Chapter \ref{ch:futureWork}.
